\documentclass{beamer}
\usepackage{beamerthemesplit} % new 
\usepackage[utf8]{inputenc}
\begin{document}
\title{Technologies mobiles} 
\author{Olivier Levitt} 
\date{\today} 



\frame{
\titlepage
\includegraphics[width=60pt]{google-android.jpg}
\includegraphics[width=60pt]{ios.png}
\includegraphics[width=60pt]{wp8.jpg}
\includegraphics[width=60pt]{bb10.jpg}
} 

\frame{\frametitle{Sommaire}\tableofcontents} 
 
\section{Présentation et objectifs du cours}
\subsection{Organisation administrative}
\frame{
\frametitle{Planning}
\begin{itemize}
  \item{30 janvier : 3h de cours, 3h de TP}
  \item{6 février : 3h de cours}
  \item{13 février : 6h de TP}
  \item{Validation des sujets de projet avant le 20 février}
  \item{20 février : 6h de TP dédiées au projet}
  \item {? mars : Soutenance du projet}
\end{itemize}
}
\frame{
\frametitle{Evaluation}
\begin{itemize}
  \item{Projet + soutenance}
  \item{Groupe de 2}
  \item{Sujet ``libre''}
  \item{6h de TP dédiées au projet + travail personnel}
\end{itemize}
}
\subsection{Champ du cours}
\frame{
\frametitle{Contexte}
\begin{itemize}
  \item Smartphones, tablettes et assimilés (TV, montre, autoradio \ldots)
  \item Dev d'application, pas de dev de la plateforme
  \item 1ère partie : le dev mobile en général
  \item 2ème partie : application sous android
\end{itemize}
}

\section{Le développement mobile} 
\subsection{Comparaison des différents OS mobile}
\subsection{Spécificités du développement mobile}
\frame{
\frametitle{Des appareils suréquipés}
\begin{itemize}
  \item Téléphonie (SMS, MMS, appels)
  \item Internet (GPRS, edge, 3g, 4g, wifi)
  \item Réseaux locaux (Bluetooth, réseaux adhoc, NFC)
  \item Capteurs (Luminosité, proximité)
  \item Localisation (GPS, triangulation, SSID wifi)
  \item Notifications (Vibreur, haut-parleurs, LED)
  \item Stockage de données (Mémoire flash, SD externe, SQLite)
  \item Interactions (Ecran tactile, gestures, boutons physique)
  \item Et encore d'autres \ldots
\end{itemize}
Et des API pour utiliser tout ça !
}
\frame{
\frametitle{Des contraintes techniques importantes}
\begin{itemize}
  \item Processeur
  \item Mémoire RAM
  \item Stockage de données
  \item Gestion de la batterie
  \item Stabilité et débit de la connexion internet
  \item Cycle de vie de l'application
  \item Taille d'écran 
  \item Inputs atypiques (clavier virtuel, gestures, peu de boutons \ldots)
\end{itemize}
Contraintes à garder en tête en permanence.
}
\frame{
\frametitle{La fragmentation}
Une application publiée sur le google playstore cible plus de 2400 appareils
différents !\\
\begin{itemize}
   \item ``Write once, run everywhere'' ?
   \item Comment tester / débugger pour tous ces appareils ?
  \item Eviter de géner l'utilisateur (versions HD, appareils non compatibles)
  \item S'adapter quand une fonctionnalité n'est pas disponible
\end{itemize}
}
\frame{
\frametitle{La fragmentation, taille d'écran}
Comment gérer toutes les tailles d'écran ? 
\begin{itemize}
  \item Montres connéctées : de 1 à 2 pouces
  \item Smartphones lowcost : 3 pouces (Galaxy pocket, galaxy Y)
  \item Smartphones high-end : 4 à 5 pouces (IPhone 5, HTC 8X, nexus 4)
  \item Phablets : 5 à 6 pouces (Galaxy note, HTC butterfly)
  \item Tablettes : 7 pouces (Nexus 7, IPad mini), 8 pouces (Archos 80g9), 10 pouces (Nexus 10, IPad)
\end{itemize}
}


\frame{
\frametitle{De nombreuses autres sources de fragmentation}
\begin{itemize}
  \item Versions de l'OS
  \item Résolutions d'écran
  \item Elements hardware présents
  \item Puissance
  \item Modifications constructeur / ``rom custom''
  \item \ldots
\end{itemize}
}
\frame{
\frametitle{Des Ecosystèmes forts}
\begin{itemize}
  \item Obligation d'utiliser le SDK fourni
  \item Suivre les guidelines
  \item Restrictions liées à la plateforme 
  \item Utilisation des services de la plateforme
  \item Processus de déploiement des applications
  \item Règles des ``store'' (validation, monétisation \ldots)
\end{itemize}
}
\section{Le développement sur android} 
\frame{\frametitle{Sommaire}\tableofcontents} 
\subsection{Mise en place}
\frame{
\frametitle{Les marque-page}
\begin{itemize}
  \item www.frandroid.com (actu FR)
  \item www.androidpolice.com (actu EN)
  \item www.androidcentral.com (actu EN)
  \item www.stackoverflow (Q/A EN)
  \item \#android et \#android-dev sur freenode (chat irc EN)
\end{itemize}
}
\frame{
\frametitle{Bien commencer}
La programmation android fait partie des plus accessibles :
\begin{itemize}
  \item Des (bonnes) bases de programmation en JAVA
  \item Un ordinateur (Windows, Linux, Mac OS X)
  \item Un appareil android (conseillé, l'émulateur étant \ldots moyen)
\end{itemize}
C'est tout et c'est gratuit !
}
\frame{
\frametitle{Téléchargement et installation du SDK}
}

\frame{
\frametitle{Distribuer l'application}
\begin{itemize}
  \item Une application android = un APK (+/- équivalent d'un jar)
  \item Distribution directe de l'APK (ex : pour tester, béta fermée)
  \item Publication sur le playstore, 25\$ à l'inscription
  \item Application gratuite ou payante (30\% pour google)
 \end{itemize}
}
\subsection{Architecture}
\frame{
\frametitle{Activity et cycle de vie de l'application}
}
\subsection{IHM}
\subsection{Données}




\end{document}
