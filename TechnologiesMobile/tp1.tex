\documentclass{article}
\usepackage[utf8]{inputenc}% encodage du fichier source
\usepackage[T1]{fontenc}% gestion des accents (pour les pdf)
\usepackage[francais]{babel}% rajouter éventuellement english, greek, etc.
\usepackage{listings}

\title{ANDROID - TD1}
\date{Source et pdf de ce TD
:\\\url{https://bitbucket.org/olevitt/technologies-mobiles}}
\maketitle



\begin{document}
\section{Découvrir le SDK}
\subsection{Verifier l'installation}
\begin{itemize}
\item Lancer Eclipse
\item Vérifier dans Windows => Preferences la présence du menu android
\item Vérifier que le chemin du SDK est bien rempli
\end{itemize}
\begin{enumerate}
\item Quels sont les niveaux d'API disponibles ?
\end{enumerate}
\subsection{Hello world}
\begin{itemize}
\item Créer un nouveau projet android via \textbf{File/New/Other/Android/Android
Project}
\item Executer l'application (Ctrl + F11)
\item Eclipse ne trouve pas d'appareil cible (normal)
\item L'android virtual device manager s'ouvre alors et permet de créer un
émulateur.
\item L'AVD manager est aussi disponible via Window => Android virtual
device manager.
\end{itemize}
\subsection{Créer un appareil virtuel}
\begin{itemize}
\item Créer un nouveau projet android via \textbf{File/New/Other/Android/Android
Project}
\item Executer l'application (Ctrl + F11)
\item Eclipse ne trouve pas d'appareil cible (normal)
\item L'android virtual device manager s'ouvre alors et permet de créer un
émulateur.
\item L'AVD manager est aussi disponible via Window => Android virtual
device manager.
\end{itemize}
\end{document}
