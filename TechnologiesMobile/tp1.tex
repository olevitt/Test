\documentclass{article}
\usepackage[utf8]{inputenc}% encodage du fichier source
\usepackage[francais]{babel}% rajouter éventuellement english, greek, etc.
\usepackage{listings}
\usepackage{hyperref}
\hypersetup{urlcolor=,linkcolor=} % Does not apply color to href's

\title{ANDROID - TP1}
\date{Source et pdf de ce TP
:\\ \href{https://bitbucket.org/olevitt/technologies-mobiles}{https://bitbucket.org/olevitt/technologies-mobiles}}
\maketitle



\begin{document}
Ce TP a deux objectifs :
\begin{itemize}
\item Se familiariser avec le SDK android et le plugin ADT pour eclipse
\item Construire une petite application de simulateur de crédit
\end{itemize}
\section{Mise en place}
\subsection{Verifier l'installation}
\begin{itemize}
\item Lancer Eclipse
\item Vérifier dans \textbf{Window/Preferences} la présence du menu android
\item Vérifier que le chemin du SDK est bien rempli
\end{itemize}
\begin{enumerate}
\item Quels sont les niveaux d'API disponibles ?
\end{enumerate}
\subsection{Hello world}
\begin{itemize}
\item Créer un nouveau projet android via \textbf{File/New/Other/Android/Android
Project}
\item Executer l'application (Ctrl + F11)
\end{itemize}
\begin{enumerate}
 \setcounter{enumi}{1}
\item Pourquoi l'application ne se lance t'elle pas ?
\end{enumerate}
\begin{itemize}
\item L'android virtual device manager s'ouvre et permet de créer un
émulateur
\item L'AVD manager est aussi disponible via \textbf{Window / Android virtual
device manager}
\end{itemize}
\subsection{Créer un appareil virtuel}
\begin{itemize}
\item Créer un nouvel appareil avec le plus grand niveau d'API possible
\item Choisir x86 plutôt qu'ARM si possible
\item Démarrer l'appareil
\item Manipuler un peu l'émulateur
\end{itemize}
\begin{enumerate}
 \setcounter{enumi}{2}
\item Quelles sont les applications pré-installées sur l'émulateur ?
\item Comment voir la log de l'émulateur ?
\end{enumerate}
\newpage
\section{Construire notre simulateur de crédit}
\subsection{Un peu d'UI}
Modifier le layout main pour y ajouter quelques composants :
\begin{itemize}
\item Un champ de texte (EditText) : montant
\item Un champ de texte (EditText) : durée
\item Ajouter un bouton (Button) : calculer
\item Obliger l'utilisateur à ne rentrer que des chiffres dans les champs en utilisant la propriété \href{http://developer.android.com/reference/android/widget/TextView.html#attr_android:inputType}{InputType}
\end{itemize}
\begin{enumerate}
 \setcounter{enumi}{4}
\item Que faut t'il prévoir pour pouvoir utiliser ces views en java ?
\end{enumerate}
\subsection{Rendre l'interface vivante}
\begin{itemize}
\item Ecouter les clicks sur le bouton et déclencher un toast lors d'un click sur le bouton
\item Rappel : 2 techniques ont été vues en cours pour faire ça
\item Dans le toast, afficher le contenu du champ de texte (editText.getText())
\end{itemize}
\subsection{Implémenter la logique métier}
\begin{itemize}
\item Lors d'un click sur le bouton, récupérer les valeurs de montant et durée
\item Aide : \href{http://developer.android.com/reference/java/lang/Integer.html#parseInt(java.lang.String)}{Integer.parseInt(String)} permet de convertir une chaîne en entier
\item Implémenter la logique métier pour calculer les mensualités (taux fixe à 4\% ou variable en fonction de la durée)
\item Afficher ce résultat d'abord dans un toast puis dans une view (TextView)
\end{itemize}
\begin{enumerate}
 \setcounter{enumi}{5}
\item Comment cacher la view contenant les mensualités tant qu'elles n'ont pas été calculées ?
\end{enumerate}
\newpage
\section{Aller plus loin}
\subsection{Partager le résultat}
\begin{itemize}
\item Ajouter un bouton : partager
\item Mettre une icône de partage à ce bouton (setBackgroundResource)
\item Lors d'un click sur le bouton, lancer un intent implicite pour partager le résultat
\end{itemize}
\begin{enumerate}
 \setcounter{enumi}{6}
\item Comment filtrer les activities proposées pour le partage ?
\item Que se passe t'il si aucune activity n'est capable de répondre à l'intent ?
\end{enumerate}
\subsection{Créer un historique des calculs}
\begin{enumerate}
 \setcounter{enumi}{8}
\item Quelle solution de stockage de l'historique des calculs vous parait la mieux adaptée ? (Rappel : on a le choix entre préferences, BDD SQLite et fichier plat) 
\end{enumerate}
\begin{itemize}
\item Mettre en place la solution choisie
\item A chaque calcul, rajouter un élément à l'historique
\end{itemize}
\subsection{Afficher l'historique}
\begin{itemize}
\item L'historique sera affiché dans un écran séparé
\item Créer une nouvelle Activity qui contiendra l'historique (penser à la déclarer dans le Manifest)
\item Quel ViewGroup vous paraît le plus adapté pour afficher l'historique (nombre indéfini d'éléments)
\begin{enumerate}
 \setcounter{enumi}{9}
\item Comment lancer cette activité ?
\end{enumerate}
\end{itemize}
\end{document}

