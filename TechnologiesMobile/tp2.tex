\documentclass{article}
\usepackage[utf8]{inputenc}% encodage du fichier source
\usepackage[francais]{babel}% rajouter éventuellement english, greek, etc.
\usepackage{listings}
\usepackage{hyperref}
\usepackage[margin=2.5cm]{geometry}
\hypersetup{urlcolor=,linkcolor=} % Does not apply color to href's
\lstset{
	tabsize=4,
	language=Java,
        basicstyle=\scriptsize,
        columns=fixed,
        extendedchars=true,
        breaklines=true,
		frame=single,
        showtabs=false,
        showspaces=false,
        showstringspaces=false,
        identifierstyle=\ttfamily,
        keywordstyle=\color[rgb]{0,0,1},
        commentstyle=\color[rgb]{0.133,0.545,0.133},
        stringstyle=\color[rgb]{0.627,0.126,0.941},
        numbers=left, 
        numberstyle=\tiny,
        xleftmargin=\parindent
}


\title{ANDROID - TP2}
\date{Source et pdf du cours et de ce TP
:\\ \href{https://bitbucket.org/olevitt/technologies-mobiles}{https://bitbucket.org/olevitt/technologies-mobiles/src}}

\begin{document}
\maketitle
L'objectif de ce TP est de réaliser une todo list (liste de ``choses à faire'').\\
Ce TP fera appel aux notions suivantes :
\begin{itemize}
  \item ListView
  \item RelativeLayout
  \item Intents
  \item Stockage de données
\end{itemize}
Il est fortement conseillé d'avoir le cours à portée de main et à ne pas hésiter à lire la documentation officielle
\href{http://d.android.com}{http://d.android.com} ainsi que la multitude de tutoriaux disponibles.\\

\section{L'écran principal : affichage des tâches}
\begin{itemize}
  \item Créer un nouveau projet android avec une activity
  \item Modifier le layout de cette activity pour y ajouter une ListView qui prendra toute la largeur et toute la hauteur
\end{itemize}
\begin{enumerate}
 \setcounter{enumi}{0}
 \item Quel composant est chargé de faire le lien entre les données et la ListActivity ? (on l'implémentera plus tard)
\item On souhaite ajouter un bouton ``ajouter une tâche'' en dessous de la ListView. Quel layout proposez vous ?
\item Pourquoi un LinearLayout avec orientation ``vertical'' ne convient-il pas ?
\end{enumerate}
\begin{itemize}
  \item Ajouter le bouton en lui faisant prendre la place nécessaire en hauteur et toute la largeur.
\end{itemize}
\section{Un écran secondaire : ajout d'une tâche}
\begin{itemize}
  \item Créer une deuxième activity (rappel : ne pas oublier de l'ajouter au manifest !)
  \item Ajouter deux champs de texte (nom et commentaires) ainsi qu'un bouton ``ajouter'' à cette activity (libre à vous de définir le layout)
\end{itemize}
\begin{enumerate}
 \setcounter{enumi}{3}
\item Cette activity n'étant pour l'instant appelable que depuis l'écran principal, quels intent-filters proposez vous ? 
\end{enumerate}
\section{Enchainement des écrans}
\subsection{Dans un sens \ldots}
\begin{itemize}
  \item Sur l'activity principale, récupérer les clicks sur le bouton ``ajouter une tâche''
 \end{itemize}
\begin{enumerate}
 \setcounter{enumi}{4}
\item On veut lancer l'activity d'ajout d'une tâche lors d'un click sur ce bouton. Quel concept android va t'on utiliser ? 
(indice : il peut être soit implicite soit explicite)
\item Dans ce cas, est-il implicite ou explicite ?
\item Pensez vous qu'il faille passer des données lors de l'appel à l'activity ``ajouter une tâche'' ? Si oui, lesquelles ?
\end{enumerate}
\begin{itemize}
  \item Lancer l'activity d'ajout de tâche lors d'un click sur le bouton.
\end{itemize}
\subsection{\ldots et dans l'autre}

\begin{itemize}
  \item Lors d'un click sur le bouton ``ajouter'' de l'activity d'ajout, on souhaite retourner à l'écran principal.
 \end{itemize}
 \begin{enumerate}
 \setcounter{enumi}{7}
\item Est-il judicieux de lancer un intent vers l'activity principale ? 
\end{enumerate}
\begin{itemize}
  \item En réalité, android construit une pile des activités lancées (stack). \href{http://developer.android.com/guide/components/tasks-and-back-stack.html}{Documentation officielle sur les stacks} 
  \item En faire l'éxpérience en lançant un intent vers l'activity principale depuis cette activity et en appuyant ensuite sur la touche retour de l'appareil. 
  \item Utiliser la méthode \href{http://developer.android.com/reference/android/app/Activity.html#finish()}{finish()} de Activity pour fermer l'activity et ainsi revenir à l'écran principal lors d'un click sur le bouton ajouter 
 \end{itemize}
 
 \section{Place aux données !}
 \begin{itemize}
  \item Créer une classe déstinée à contenir les données : Item. Chaque élément de la todo list contiendra au moins un nom et une description.
 \end{itemize}
 \subsection{Afficher les données dans la ListView}
   \begin{enumerate}
 \setcounter{enumi}{8}
\item Au vu de la simplicité des données à afficher pour chque élément (2 champs de texte), quel type d'adapter vous parait le plus intéressant ? 
(Rappel : on a le choix entre ArrayAdapter, CursorAdapter et faire sa propre implémentation à partir de BaseAdapter)
\end{enumerate}
 \begin{itemize}
  \item Implémenter la solution choisie en utilisant un jeu de de données de test (initialiser les données à la volée)
 \end{itemize}
  \subsection{Stocker les données}
  \begin{enumerate}
 \setcounter{enumi}{9}
\item Quelle solution de stockage des données vous parait le plus adaptée ? (Rappel : on a le choix entre Preferences, fichier plat, base de données SQLite et stockage distant)
\end{enumerate}
 \begin{itemize}
  \item Pour des raisons de simplicité, on va opter dans la suite pour un stockage en fichier plat (libre à vous de le remplacer par une base SQLite)
  \item Créer une méthode qui écrit une liste d'éléments Item dans un fichier (utiliser un format de stockage primitif par exemple nom#description et un item par ligne. C'est un mauvais choix en pratique car peu robuste et peu extensible mais ça suffira largement ici).
  \item Créer une méthode qui instancie une liste d'éléments Item à partir d'un fichier.
  \item Dans la méthode onCreate(), remplir la ListView à partir des données contenues dans le fichier.
 \end{itemize}
 \subsection{Remplir les données}
 \begin{itemize}
  \item Lors d'un click sur le bouton ``ajouter'' de l'activity d'ajout, créer une instance d'Item et la rajouter au fichier. (pour simplifier, réecrire la totalité du fichier à chaque fois)
 \end{itemize}
  \begin{enumerate}
 \setcounter{enumi}{10}
\item Pourquoi la ListView n'est elle pas mise à jour lors du retour sur l'activity principale ?
\end{enumerate}
 \begin{itemize}
  \item Utiliser le cycle de vie des activities pour recharger les données de la ListView lors du retour sur l'activity principale. (implémenter la bonne méthode onXXXXX)
 \end{itemize}
 \section{Voir le détail d'un item}
 \begin{itemize}
  \item Créer une activity de visualisation d'item avec les champs utiles (nom, description \ldots)
  \item Récupérer les clicks sur les éléments de la ListView (Attention à bien utiliser onItemClickListener et non onClickListener).
  \item Lancer l'activity de visualisation lors d'un click sur un item de la liste
  \item Transmettre à l'activity soit l'identifiant de l'item soit les données de l'item
  \item Récupérer les données dans l'activity de visualisation et afficher les données de l'item
 \end{itemize}
 \section{Améliorations possibles}
	Wow, déjà tout fait ? Voici quelques pistes d'améliorations
  \begin{itemize}
  \item Le stockage en fichier plat n'est pas très judicieux. Mettre en place une base de données SQLite à la place.
  \item Les données associées aux items sont pour l'instant minimales, pourquoi ne pas ajouter un identifiant, une date d'ajout, un statut (en cours, fait \ldots), une priorité, une catégorie \ldots ?
  \item L'affichage des éléments dans la liste est basique mais on peut difficilement faire mieux avec ArrayAdapter. Créer un adapter maison en héritant de BaseAdapter
  \item Partager les items par mail, sms, twitter \ldots en ajoutant un intent implicite dans l'activity de visualisation ou de modification.
 \end{itemize}
\end{document}


