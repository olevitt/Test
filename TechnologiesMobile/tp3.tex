\documentclass{article}
\usepackage[utf8]{inputenc}% encodage du fichier source
\usepackage[francais]{babel}% rajouter éventuellement english, greek, etc.
\usepackage{listings}
\usepackage{hyperref}
\usepackage[margin=2.5cm]{geometry}
\hypersetup{urlcolor=,linkcolor=} % Does not apply color to href's
\lstset{
	tabsize=4,
	language=Java,
        basicstyle=\scriptsize,
        columns=fixed,
        extendedchars=true,
        breaklines=true,
		frame=single,
        showtabs=false,
        showspaces=false,
        showstringspaces=false,
        identifierstyle=\ttfamily,
        keywordstyle=\color[rgb]{0,0,1},
        commentstyle=\color[rgb]{0.133,0.545,0.133},
        stringstyle=\color[rgb]{0.627,0.126,0.941},
        numbers=left, 
        numberstyle=\tiny,
        xleftmargin=\parindent
}


\title{ANDROID - TP3}
\date{Source et pdf du cours et de ce TP
:\\ \href{https://bitbucket.org/olevitt/technologies-mobiles}{https://bitbucket.org/olevitt/technologies-mobiles/src}}

\begin{document}
\maketitle
L'objectif de ce TP est de réaliser une application utilisant les données de la star pour informer l'utilisateur des horaires de bus.\\
Ce TP fera appel aux notions suivantes :
\begin{itemize}
  \item Requêtes HTTP (webservice)
  \item Multithreading
  \item Parsing JSON / XML
  \item Stockage de données SQLite
\end{itemize}
Il est fortement conseillé d'avoir le cours à portée de main et de ne pas hésiter à lire la documentation officielle 
\href{http://d.android.com}{http://d.android.com} ainsi que la multitude de tutoriaux disponibles.\\
\section{Présentation de l'API star}
La star a fait de gros efforts d'ouverture des données (open data) ce qui permet à tous les développeurs d'utiliser les données dans leurs applications (qu'elles soient libres ou fermées, gratuites ou payantes, avec ou sans pub \ldots)\\
Les données disponibles sont nombreuses, entre autres les horaires des bus, la géolocalisation des arrêts, les infos trafic et même le nombre de vélos disponibles dans chaque station de vélostar.\\
La liste des données disponibles est consultable à cette adresse : \href{http://data.keolis-rennes.com/fr/les-donnees/donnees-et-api.html}{http://data.keolis-rennes.com/fr/les-donnees/donnees-et-api.html}}
\end{document}


