\documentclass{article}
\usepackage[utf8]{inputenc}% encodage du fichier source
\usepackage[francais]{babel}% rajouter éventuellement english, greek, etc.
\usepackage{listings}
\usepackage{hyperref}
\usepackage[margin=2.5cm]{geometry}
\hypersetup{urlcolor=,linkcolor=} % Does not apply color to href's
\lstset{
	tabsize=4,
	language=Java,
        basicstyle=\scriptsize,
        columns=fixed,
        extendedchars=true,
        breaklines=true,
		frame=single,
        showtabs=false,
        showspaces=false,
        showstringspaces=false,
        identifierstyle=\ttfamily,
        keywordstyle=\color[rgb]{0,0,1},
        commentstyle=\color[rgb]{0.133,0.545,0.133},
        numbers=left, 
        numberstyle=\tiny,
        xleftmargin=\parindent
}


\title{ANDROID - TP3}
\date{Source, pdf et corrigé de ce TP
:\\\href{http://tiny.cc/techmob}{http://tiny.cc/techmob}}

\begin{document}
\maketitle
L'objectif de ce TP est de réaliser une application utilisant les données
publiques des transports en commun de rennes afin d'afficher le temps
d'attente des bus.\\
Ce TP fera appel aux notions suivantes :
\begin{itemize}
  \item Réseau / requêtes HTTP (webservice)
  \item Parsing JSON
  \item Stockage de données SQLite
  \item Multithreading
  \item ListView
\end{itemize}
Il est fortement conseillé d'avoir le cours à portée de main et à ne pas hésiter
à se référer à la documentation officielle
\href{http://d.android.com}{http://d.android.com} ainsi que la multitude de tutoriaux disponibles.\\
\section{Présentation des données star}
\subsection{Contexte}
Rennes a fait partie des premières villes à libérer ses données publiques
concernant les transports en commun. Ainsi, la star met à disposition des
développeurs une grande partie des données concernant son réseau (stations,
horaires, prochains passages \ldots).\\
Les données sont disponibles pour tous, gratuitement.
\subsection{Contenu des données}
Les données disponibles sont réparties en deux catégories : les données
statiques et les données temps réel.
\subsubsection{Les données statiques}
Les données statiques contiennent les informations sur le réseau star : les
stations et leurs emplacements, les lignes et leurs horaires, les équipements
disponibles en station \ldots\\
Ces données n'évoluent que quelques fois par an.\\
La star publie ses données au standard \href{https://developers.google.com/transit/gtfs/reference}{GTFS (General
Transit Feed Specification)}. Concretement, il s'agit simplement de plusieurs
fichiers csv.\\
La documentation sur ces données et la dernière version des données sont
disponibles ici :
\href{http://data.keolis-rennes.com/fr/les-donnees/donnees-telechargeables.html}{http://data.keolis-rennes.com/fr/les-donnees/donnees-telechargeables.html)}

\subsubsection{Les données temps réel}
La star met aussi à disposition des données temps réel. On y trouve le nombre de
vélos et de places disponibles à une station vélo star, les heures de prochain
passage des bus et métro \ldots\\
La liste des données disponibles est disponible ici : 
\href{http://data.keolis-rennes.com/fr/les-donnees/donnees-et-api.html}{http://data.keolis-rennes.com/fr/les-donnees/donnees-et-api.html}\\
L'accès à ces données se fait via un webservice.

\subsection{Principe d'un webservice}
Un webservice est, comme son nom l'indique, un service accessible sur le web.\\
Techniquement, il s'agit d'un programme tournant sur un serveur et déstiné à
diffuser des données brutes (contrairement au HTML, il n'y a pas d'informations
sur la façon d'afficher ces données) via le protocole HTTP.\\
Toutes les requêtes sont testables directement et tapant l'URL dans votre
navigateur préféré.
\subsection{Appel au webservice}
Un webservice est toujours constitué d'une adresse de base suivi d'un paramètre
indiquant le type d'information demandée et éventuellement de paramètres
précisant cette information.\\
Pour le webservice star, l'URL de base est la suivante : 
\href{http://data.keolis-rennes.com/json/}{http://data.keolis-rennes.com/json/}\\
Il y a ensuite 3 paramètres obligatoires :
\begin{itemize}
  \item cmd : la commande demandée. C'est à dire, l'information que l'on
  cherche. Par exemple, pour obtenir le prochain départ d'un bus, on a
  $cmd=getbusnextdepartures$
  \item version : la version du webservice que l'on utilise. Se référer à la
  documentation pour savoir quelle version utiliser pour chaque commande. 
  \item key : la clé. Afin de savoir qui utilise le webservice, la star fournit
  aux développeurs des clés correspondant à chaque application déclarée. Pour
  obtenir une clé, il faut s'inscrire sur le site. Pour tout ce TP, vous pouvez
  utiliser la clé suivante : 1RJLZ38TUFZSWTW
\end{itemize}
Ainsi, un appel au webservice peut se faire directement à l'URL suivante :\\
\href{http://data.keolis-rennes.com/json/?cmd=getmetrostatus&version=2.2&key=1RJLZ38TUFZSWTW}{http://data.keolis-rennes.com/json/?cmd=getmetrostatus&version=2.2&key=1RJLZ38TUFZSWTW}
\subsection{Résultat du webservice}
Le webservice renvoie les données au format JSON ou au format XML (changer
JSON en XML dans l'URL).\\
Pour simplifier, dans ce TP, on ne manipulera que les données JSON.
\subsection{Visualiser le JSON} 
Même si le format JSON est relativement clair et simple, il reste assez
difficile d'obtenir une vision globale des données présentes dans un résultat JSON d'un seul coup d'oeil.\\
Des outils de visualisation JSON existent et sont très utiles dès lors que l'on
manipule des fichiers JSON.\\
Vous pouvez par exemple utiliser
\href{http://jsonviewer.stack.hu/}{http://jsonviewer.stack.hu/}
\section{Hello World !}
\begin{itemize} 
  \item Créer un nouveau projet (EmptyActivity)
  \item Ajouter la permission INTERNET au manifest
  \item Dans le onCreate de votre Activity, ajouter la ligne suivante 
\begin{lstlisting}[language=XML]
StrictMode.setThreadPolicy(new ThreadPolicy.Builder().permitAll().build());
\end{lstlisting}
Cette ligne sera expliquée plus tard dans le TP
  \item Utiliser
  \href{http://developer.android.com/reference/java/net/HttpURLConnection.html}{HttpURLConnection}
  pour faire une requête HTTP vers le webservice
\end{itemize}
Pour lire le contenu d'un InputStream (flux entrant), on peut utiliser le code
suivant :
 \begin{lstlisting}[language=XML]
 public String readStream(InputStream inputStream) throws IOException {
     BufferedReader reader = new BufferedReader(new
     InputStreamReader(inputStream)); 
     String ligne = null;
     String contenu = "";
     while ((ligne = reader.readLine()) != null) {
         contenu += ligne;
     }
     return contenu;
 }
\end{lstlisting}
\begin{itemize} 
  \item Afficher dans la log le contenu de la réponse.
\end{itemize}
\section{Parsing des données}
\subsection{Lire le JSON}
\begin{itemize} 
  \item Créer un objet
  \href{http://developer.android.com/reference/org/json/JSONObject.html}{JSONObject}
  à partir des données récupérées précedemment
  \item En utilisant la méthode
  \href{http://developer.android.com/reference/org/json/JSONObject.html#getJSONArray(java.lang.String)}{getJSONArray},
  récupérer le tableau des cours  
  \item Afficher, dans un Toast, le nombre de cours  
\end{itemize}
\subsection{Instancier les objets correspondants}
\begin{itemize} 
  \item Créer une classe JAVA ``Cours" représentant un cours de l'emploi du
  temps
  \item Donner à cette classe les mêmes attributs que les données disponibles dans la réponse JSON
  \item Instancier, à partir du JSONObject, une liste (ou un tableau) d'objet
  JAVA ``Cours''
\end{itemize}
\section{Stockage des données}
L'objectif est de stocker les données chargées pour éviter de devoir les
retélécharger à chaque fois (gain de temps, accès hors connexion, économie de
batterie \ldots)
\begin{enumerate}
 \setcounter{enumi}{0}
\item Quelle solution de stockage vous parait la plus adaptée ?
\end{enumerate}
\begin{itemize} 
  \item Implémenter le nécessaire pour créer une base de données SQLite
  (utiliser SQLiteOpenHelper)
  \item Dans onCreate, éxecuter une requête de création de table pour créer
  une table Cours avec tous les attributs de la classe Cours 
  \item Exécuter une insertion avec des données fictives
  \item Exécuter une ``query'' pour récupérer les données stockées et instancier
  les objets ``Cours'' correspondant aux données, les afficher dans la log
  \item Insérer les données récupérées via le webservice
\end{itemize}
\begin{enumerate}
 \setcounter{enumi}{1}
\item Imaginons qu'un jour pamplemousse change et fournisse des
données en plus.\\
Comment modifier la base de données SQLite si elle a déjà été crée sur l'appareil ?
\end{enumerate}
\section{Il est temps de faire les choses proprement}
\begin{enumerate}
 \setcounter{enumi}{2}
\item Quel point, vu dans les bonus, a été complétement ignoré depuis le début de ce TP ?
\end{enumerate}
\begin{itemize}
  \item Remplacer la ligne StrictMode ajoutée au début du TP par celle ci
 \begin{lstlisting}[language=XML]
 StrictMode.setThreadPolicy(new ThreadPolicy.Builder().detectNetwork().penaltyDeathOnNetwork().build());
\end{lstlisting} 
  \item Relancer l'application, pleurer
  \item
  \href{http://developer.android.com/reference/android/os/StrictMode.html}{StrictMode}
  est un outil d'aide au développement permettant de détecter les opérations
  longues succeptibles de bloquer un Thread.
  \item Le code ci-dessus indique à StrictMode de faire crasher l'application
  dès qu'un appel réseau est fait dans le Thread actif (ici, le Thread principal
  : main / UI). Depuis le niveau d'API 11, c'est le comportement par défaut.\\La
  ligne ajoutée en début de TP disait à StrictMode de tout laisser passer.
  \item Déplacer le bloc de code correspondant à la requête HTTP dans un Thread
  séparé (utiliser new Thread(Runnable)), sourire.
\end{itemize}
\section{Afficher les données}
\subsection{Afficher la liste des cours}
\begin{enumerate}
 \setcounter{enumi}{3}
\item Quel view / viewgroup vous parait le plus adapté pour afficher les cours ?
\end{enumerate}
\begin{itemize} 
  \item Ajouter cet élément au layout en lui faisant prendre toute la place en
  largeur / hauteur
  \item Il nous faut maintenant un adapter : créer une nouvelle classe
  ``CoursAdapter'' qui extends BaseAdapter
  \item Hériter de BaseAdapter implique d'écrire 4 méthodes
  \item Avant d'implémenter ces méthodes, construire un constructeur prenant un
  Context et une List de Cours et stocker ces valeurs dans des attributs de
  classe
 \end{itemize}
 \begin{description}
   \item getCount : renvoyer le nombre d'éléments de la liste
   \item getItem : renvoyer le n-ième élément de la liste
   \item getItemId : renvoyer position (il suffit que l'id de chaque item soit
   unique)
   \item getView : ci-dessous un squelette d'une méthode getView. Il faut au
   préalable créer un layout ``layoutcours'' représentant un élement de la
   liste.
 \end{description}
  \begin{lstlisting}[language=XML]
   public View getView(int position, View convertView, ViewGroup parent) {
			View view;
			
			if (convertView == null) {
				view = ((LayoutInflater) context.getSystemService(Context.LAYOUT_INFLATER_SERVICE)).inflate(R.layout.layoutcours, parent, false);
			} else {
				view = convertView;
			}
			Cours cours = (Cours) getItem(position);
	
			TextView nomCours = (TextView) view.findViewById(R.id.nomcours);
			nomCours.setText(cours.getNom());
			
			return view;
		}
\end{lstlisting} 

 \begin{itemize} 
  \item Instancier un CoursAdapter et l'affecter à la ListView via
  listView.setAdapter
  \item Il est possible d'affiner le layout de chaque item en rajoutant des
  champs et d'autres informations. Rajouter la date de début du cours.
\end{itemize}
\subsection{Afficher le détail d'un cours}
\begin{itemize} 
  \item Créer une activity de visualisation d'un cours en particulier
  \item En utilisant le bon listener, récupérer les clicks faits sur un élément
  de la ListView
  \item Lors d'un click sur un élément, ouvrir l'activity de visualisation
\end{itemize}
\begin{enumerate}
 \setcounter{enumi}{5}
 \item Il y a t'il des paramètres à passer à l'activity de visualisation ?
 \end{enumerate}
 \begin{itemize} 
  \item Rajouter dans l'intent les données nécessaires à l'activity de
  visualisation
  \item Dans le onCreate de l'activity de visualisation, récupérer les données
  de l'intent
  \item A partir des données de l'intent, récupérer en base les données
  correspondant à ce cours
  \item Afficher ces données (vous êtes libres du layout)
\end{itemize}
\section{Pour aller plus loin}
  Wow, déjà tout fait ? Voici quelques pistes d'améliorations :
\begin{itemize}
  \item Ajouter un bouton dans l'activity principale pour recharger les données
  \item Créer un écran de gestion des préférences pour que l'utilisateur puisse
  renseigner son identifiant / mot de passe
  \item Les bonnes pratiques (guidelines) recommandent d'informer l'utilisateur lors des chargements et en particulier lors de l'accès au réseau. Implémenter un ProgressDialog pour informer l'utilisateur des chargements.
 \end{itemize}
\end{document}


