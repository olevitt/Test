\documentclass{article}
\usepackage[utf8]{inputenc}% encodage du fichier source
\usepackage[francais]{babel}% rajouter éventuellement english, greek, etc.
\usepackage{listings}
\usepackage{hyperref}
\usepackage[margin=2.5cm]{geometry}
\hypersetup{urlcolor=,linkcolor=} % Does not apply color to href's
\lstset{
	tabsize=4,
	language=Java,
        basicstyle=\scriptsize,
        columns=fixed,
        extendedchars=true,
        breaklines=true,
		frame=single,
        showtabs=false,
        showspaces=false,
        showstringspaces=false,
        identifierstyle=\ttfamily,
        keywordstyle=\color[rgb]{0,0,1},
        commentstyle=\color[rgb]{0.133,0.545,0.133},
        numbers=left, 
        numberstyle=\tiny,
        xleftmargin=\parindent
}


\title{ANDROID - TP3}
\date{Source, pdf et corrigé de ce TP
:\\\href{http://tiny.cc/techmob}{http://tiny.cc/techmob}}

\begin{document}
\maketitle
L'objectif de ce TP est de réaliser une application utilisant les données
publiques des transports en commun de rennes afin d'afficher le temps
d'attente des bus.\\
Ce TP fera appel aux notions suivantes :
\begin{itemize}
  \item Réseau / requêtes HTTP (webservice)
  \item Parsing JSON
  \item Multithreading
  \item Localisation GPS
\end{itemize}
Il est fortement conseillé d'avoir le cours à portée de main et à ne pas hésiter
à se référer à la documentation officielle
\href{http://d.android.com}{http://d.android.com} ainsi qu'à la multitude de
tutoriaux disponibles.
\section{Présentation des données STAR}
\subsection{Contexte}
Rennes a fait partie des premières villes à libérer ses données publiques
concernant les transports en commun. Ainsi, la STAR met à disposition des
développeurs une grande partie des données concernant son réseau (stations,
horaires, prochains passages \ldots).\\
Les données sont disponibles pour tous, gratuitement.
\subsection{Contenu des données}
Les données disponibles sont réparties en deux catégories : les données
statiques et les données temps réel.
\subsubsection{Les données statiques}
Les données statiques contiennent les informations sur le réseau STAR : les
stations et leurs emplacements, les lignes et leurs horaires, les équipements
disponibles en station \ldots\\
Ces données n'évoluent que quelques fois par an.\\
La STAR publie ses données au standard \href{https://developers.google.com/transit/gtfs/reference}{GTFS (General
Transit Feed Specification)}. Concretement, il s'agit simplement de plusieurs
fichiers csv.\\
La documentation sur ces données et la dernière version des données sont
disponibles ici :
\href{http://data.keolis-rennes.com/fr/les-donnees/donnees-telechargeables.html}{http://data.keolis-rennes.com/fr/les-donnees/donnees-telechargeables.html)}

\subsubsection{Les données temps réel}
La STAR met aussi à disposition des données temps réel. On y trouve le nombre de
vélos et de places disponibles à une station vélostar, les heures de prochain
passage des bus et métro \ldots\\
La liste des données disponibles est disponible ici : 
\href{http://data.keolis-rennes.com/fr/les-donnees/donnees-et-api.html}{http://data.keolis-rennes.com/fr/les-donnees/donnees-et-api.html}\\
L'accès à ces données se fait via un webservice.\\
\textbf{Dans un premier temps, nous ne nous occuperons que des données
statiques.}
\section{Hello World !}
\begin{itemize} 
  \item Créer un nouveau projet
  \item Télécharger les dernières données statiques disponibles sur le site
  open-data de keolis
  \item A l'intérieur du zip téléchargé, ouvrir les fichiers routes.txt et
  stops.txt.
  Que contiennent t-ils ?
  \item Inclure ces deux fichiers dans les ressources de l'application
  \item Créer un objet métier, Route, dont les attributs reflètent le contenu du
  fichier routes (le fichier stops sera exploité plus tard)
\end{itemize}
On souhaite maintenant écrire le code permettant d'instancier les objets Route à
partir des informations présentes dans le fichier.\\
L'analyse (parsing) d'un fichier CSV est relativement facile mais peut se
réveler fastidieuse. Pour éviter de réécrire à chaque application le même code,
on peut utiliser une des multiples bibliothèques disponibles : commons-csv de la
fondation apache.\\
La documentation est disponible sur ce site : 
\href{https://commons.apache.org/proper/commons-csv/}{https://commons.apache.org/proper/commons-csv/}
\begin{itemize}
  \item Ajouter cette bibliothèque à l'application en utilisant la dépendance
  gradle : 'org.apache.commons:commons-csv:1.2'
\end{itemize}
Pour itérer sur chaque ligne du fichier, on pourra s'inspirer du code suivant
:\\
\begin{lstlisting}[language=XML]
InputStream stream = context.getResources().openRawResource(R.raw.routes);
Iterator<CSVRecord> iterateur = new CSVParser(new InputStreamReader(stream), CSVFormat.DEFAULT).iterator();
CSVRecord enregistrementCourant = null;
while (iterateur.hasNext()) {
    enregistrementCourant = iterateur.next();
}
stream.close();
\end{lstlisting}
\begin{itemize}
  \item Créer une classe DAO, RouteDAO, qui instancie une liste de Route à
  partir du contenu du fichier
  \item Afficher un toast, sur l'activity principale, indiquant le nombre de
  lignes présentes sur le réseau STAR
\end{itemize}
\section{Un peu d'interface}
\begin{itemize} 
  \item Ajouter une ListView prenant toute la taille de l'écran à l'activity
  principale
  \item Afficher, dans cette ListView, la liste des lignes (au moins le nom)
  \item Créer une nouvelle activity qui contiendra le détail des prochains
  passages des bus de cette ligne (on implémentera son contenu plus tard)
  \item Lors d'un click sur un élément de la liste des lignes, lancer l'activity
  en passant en paramètre l'identifiant de la route choisie
\end{itemize}
\section{Utilisation des données temps réel}
On va maintenant utiliser les données temps réel pour récupérer les horaires des
prochains passages pour la ligne choisie.
\subsection{Principe d'un webservice}
Un webservice est, comme son nom l'indique, un service accessible sur le web.\\
Techniquement, il s'agit d'un programme tournant sur un serveur et déstiné à
diffuser des données brutes (contrairement au HTML, il n'y a pas d'informations
sur la façon d'afficher ces données) via le protocole HTTP.\\
Toutes les requêtes sont testables directement en tapant l'URL dans votre
navigateur préféré.
\subsection{Appel au webservice}
Un webservice est toujours constitué d'une adresse de base suivi d'un paramètre
indiquant le type d'information demandée et éventuellement de paramètres
précisant cette information.\\
Pour le webservice STAR, l'URL de base est la suivante : 
\href{http://data.keolis-rennes.com/json/}{http://data.keolis-rennes.com/json/}\\
Il y a ensuite 3 paramètres obligatoires :
\begin{itemize}
  \item cmd : la commande demandée. C'est à dire, l'information que l'on
  cherche. Par exemple, pour obtenir le prochain départ d'un bus, on a
  $cmd=getbusnextdepartures$
  \item version : la version du webservice que l'on utilise. Se référer à la
  documentation pour savoir quelle version utiliser pour chaque commande. 
  \item key : la clé. Afin de savoir qui utilise le webservice, la STAR fournit
  aux développeurs des clés correspondant à chaque application déclarée. Pour
  obtenir une clé, il faut s'inscrire sur le site. Pour tout ce TP, vous pouvez
  utiliser la clé suivante : 1RJLZ38TUFZSWTW
\end{itemize}
Les autres paramètres (subparams) doivent être entourés par param[].
\\Par exemple : param[direction]=1
\\\\Ainsi, un appel au webservice pour récupérer l'état du métro (panne,
interruption, fermeture \ldots) peut se faire directement à l'URL suivante :\\
\href{http://data.keolis-rennes.com/json/?cmd=getmetrostatus&version=2.2&key=1RJLZ38TUFZSWTW}{http://data.keolis-rennes.com/json/?cmd=getmetrostatus\&version=2.2\&key=1RJLZ38TUFZSWTW}
\subsection{Résultat du webservice}
Le webservice renvoie les données au format JSON ou au format XML (changer
JSON en XML dans l'URL).\\
Pour simplifier, dans ce TP, on ne manipulera que les données JSON.
\subsection{Visualiser le JSON} 
Même si le format JSON est relativement clair et simple, il reste assez
difficile d'obtenir une vision globale des données présentes dans un résultat JSON d'un seul coup d'oeil.\\
Des outils de visualisation JSON existent et sont très utiles dès lors que l'on
manipule des fichiers JSON.\\
Vous pouvez par exemple utiliser
\href{http://jsonviewer.stack.hu/}{http://jsonviewer.stack.hu/} ou la multitude
d'extensions chrome / firefox disponibles.
\subsection{Intégration dans l'application} 
On souhaite afficher, dans la nouvelle Activity, la liste des arrêts de la ligne
choisie.
\begin{itemize} 
  \item Dans la nouvelle activity, récupérer le numéro de la ligne concernée. Si
  ce numéro n'est pas défini, afficher un toast et fermer l'activity.
  \item Ajouter la permission (uses-permission) INTERNET au manifest
  \item Dans le onCreate de votre Activity, ajouter la ligne suivante 
\begin{lstlisting}[language=XML]
StrictMode.setThreadPolicy(new ThreadPolicy.Builder().permitAll().build());
\end{lstlisting}
Cette ligne sera expliquée plus tard dans le TP
  \item Construire l'adresse qui sera appelée. La commande utilisée est
  ``getbusnextdepartures'' en mode line.
  \item Tester l'adresse dans un navigateur.
  \item Utiliser
  \href{http://developer.android.com/reference/java/net/HttpURLConnection.html}{HttpURLConnection} pour faire une requête HTTP vers le webservice
\end{itemize}
Pour lire le contenu d'un InputStream (flux entrant), on peut s'inspirer du code
suivant :
 \begin{lstlisting}[language=XML]
 public String readStream(InputStream inputStream) throws IOException {
     BufferedReader reader = new BufferedReader(new
     InputStreamReader(inputStream)); 
     String ligne = null;
     String contenu = "";
     while ((ligne = reader.readLine()) != null) {
         contenu += ligne;
     }
     return contenu;
 }
\end{lstlisting}
\begin{itemize} 
  \item Afficher dans la log le contenu de la réponse.
\end{itemize}
\section{Parsing des données}
\subsection{Lire le JSON}
\begin{itemize} 
  \item Créer un objet
  \href{http://developer.android.com/reference/org/json/JSONObject.html}{JSONObject}
  à partir des données récupérées précedemment
  \item En utilisant une combinaison des méthodes
  \href{http://developer.android.com/reference/org/json/JSONObject.html#getJSONObject(java.lang.String)}{getJSONObject}
  et 
  \href{http://developer.android.com/reference/org/json/JSONObject.html#getJSONArray(java.lang.String)}{getJSONArray}, récupérer le tableau des arrêts déservis par la ligne
  \item Afficher, dans un Toast, le nombre d'arrêts
\end{itemize}
\subsection{Instancier les objets correspondants}
\begin{itemize} 
  \item Créer des classes Java ``Arret'' et ``Departure''
  \item Donner à ces classes les mêmes attributs que les données disponibles
  dans la réponse JSON
  \item Instancier, à partir du JSONObject, une liste (ou un tableau) d'objet
  Arret
\end{itemize}
\section{Il est temps de faire les choses proprement}
\begin{enumerate}
\item Quel point, vu dans les bonus, a été complétement ignoré depuis le début de ce TP ?
\end{enumerate}
\begin{itemize}
  \item Remplacer la ligne StrictMode ajoutée au début du TP par celle ci
 \begin{lstlisting}[language=XML]
 StrictMode.setThreadPolicy(new ThreadPolicy.Builder().detectNetwork().penaltyDeathOnNetwork().build());
\end{lstlisting} 
  \item Relancer l'application, pleurer
  \item
  \href{http://developer.android.com/reference/android/os/StrictMode.html}{StrictMode}
  est un outil d'aide au développement permettant de détecter les opérations
  longues succeptibles de bloquer un Thread.
  \item Le code ci-dessus indique à StrictMode de faire crasher l'application
  dès qu'un appel réseau est fait dans le Thread actif (ici, le Thread principal
  : main / UI). Depuis le niveau d'API 11, c'est le comportement par défaut.\\La
  ligne ajoutée en début de TP disait à StrictMode de tout laisser passer.
  \item Déplacer le bloc de code correspondant à la requête HTTP dans un Thread
  séparé, sourire.
\end{itemize}
\section{Afficher les données}
\subsection{Afficher la liste des arrêts}
\begin{itemize} 
  \item Ajouter une ListView au layout
  \item Il nous faut maintenant un adapter : créer une nouvelle classe
  ``ArretAdapter'' qui extends BaseAdapter
  \end{itemize}
Hériter de BaseAdapter implique d'écrire 4 méthodes
  \begin{itemize} 
  \item Avant d'implémenter ces méthodes, construire un constructeur prenant un
  Context et une liste d'Arret et stocker ces valeurs dans des attributs de
  classe
 \end{itemize}
 \begin{description}
   \item getCount : renvoyer le nombre d'éléments de la liste
   \item getItem : renvoyer le n-ième élément de la liste
   \item getItemId : renvoyer la position (la contrainte sur ItemId est que
   chaque id soit unique)
   \item getView : ci-dessous un squelette d'une méthode getView. Il faut au
   préalable créer un layout ``layoutarret'' représentant un élement de la
   liste.
 \end{description}
  \begin{lstlisting}[language=XML]
public View getView(int position, View convertView, ViewGroup parent) {
	View view;
	
	if (convertView == null) {
		view = ((LayoutInflater)
		context.getSystemService(Context.LAYOUT_INFLATER_SERVICE)).inflate(R.layout.layoutarret
		,parent, false); 
	} 
	else { 
		view = convertView;
	}
	Arret arret = (Arret) getItem(position);

	TextView idArret = (TextView) view.findViewById(R.id.arret);
	idArret.setText(arret.getIdArret());
	
	return view;
}
\end{lstlisting} 
 \begin{itemize} 
  \item Utiliser cet ArretAdapter pour afficher les données venant du webservice
  (identifiant de l'arrêt, heure de prochain passage)
\end{itemize}
Optionnel : les informations dont le nom des arrêts sont disponibles dans le
fichier stops.txt.
\section{Station de métro la plus proche}
Le webservice donne accès à bien plus d'informations, on peut par exemple
trouver toutes les stations de métro ou de vélostar à proximité à partir de la
latitude et la longitude.\\\\
Pendant les tests, il est possible de truquer la position du GPS de l'émulateur
ou d'un appareil à partir de la vue DDMS.\\
Une multitude de sites permettent de récupérer la latitude / longitude d'un
endroit dont par exemple google maps.
\begin{itemize} 
  \item Utiliser la commande $getmetrostations$ pour récupérer les arrêts de
  métro les plus proches d'une position de votre choix
  \item Ajouter la permission ACCESS\_FINE\_LOCATION au manifest 
\end{itemize}
En s'inspirant du code suivant, récupérer la position GPS et chercher les
stations de métro les plus proches.
\begin{lstlisting}[language=XML]
LocationManager locationManager = (LocationManager) getSystemService(Context.LOCATION_SERVICE);
LocationListener locationListener = new LocationListener() {
		
		@Override
		public void onStatusChanged(String provider, int status, Bundle extras) {
			
		}
		
		@Override
		public void onProviderEnabled(String provider) {
			
		}
		
		@Override
		public void onProviderDisabled(String provider) {
		
		}
		
		@Override
		public void onLocationChanged(Location location) {
			//La localisation a changé
		}
	};
locationManager.requestLocationUpdates(
LocationManager.GPS_PROVIDER, 5000, 10, locationListener);
\end{lstlisting}
Une icône dans la barre d'état indique quand le service de localisation
  est utilisé. Afin de réduire au minimum la consommation de batterie, il est
  important de libérer les ressources quand elles ne sont plus utilisées. 
\begin{itemize} 
  \item En utilisant le cycle de vie d'une activity et la méthode
  \href{http://developer.android.com/reference/android/location/LocationManager.html#removeUpdates(android.location.LocationListener)}{removeUpdates},
  libérer les ressources lors de la fermeture de l'activity
\end{itemize}
\section{Pour aller plus loin}
\begin{itemize}
  \item Les chargements de données pouvant être longs, rajouter des popups ou
  des icônes indiquant que l'application charge des données. Eventuellement,
  afficher l'avancement du chargement.
  \item Plutôt que de lire à chaque fois les fichiers de données statiques, il
  est préférable, au lancement de l'application, de créer une base de données
  avec le contenu des fichiers routes et stops. Créer cette base et en profiter
  pour afficher le nom des arrêts dans l'activity des horaires.
\end{itemize}
\end{document}


